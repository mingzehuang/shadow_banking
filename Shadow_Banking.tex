% Options for packages loaded elsewhere
\PassOptionsToPackage{unicode}{hyperref}
\PassOptionsToPackage{hyphens}{url}
%
\documentclass[
  ignorenonframetext,
]{beamer}
\usepackage{pgfpages}
\setbeamertemplate{caption}[numbered]
\setbeamertemplate{caption label separator}{: }
\setbeamercolor{caption name}{fg=normal text.fg}
\beamertemplatenavigationsymbolsempty
% Prevent slide breaks in the middle of a paragraph
\widowpenalties 1 10000
\raggedbottom
\setbeamertemplate{part page}{
  \centering
  \begin{beamercolorbox}[sep=16pt,center]{part title}
    \usebeamerfont{part title}\insertpart\par
  \end{beamercolorbox}
}
\setbeamertemplate{section page}{
  \centering
  \begin{beamercolorbox}[sep=12pt,center]{part title}
    \usebeamerfont{section title}\insertsection\par
  \end{beamercolorbox}
}
\setbeamertemplate{subsection page}{
  \centering
  \begin{beamercolorbox}[sep=8pt,center]{part title}
    \usebeamerfont{subsection title}\insertsubsection\par
  \end{beamercolorbox}
}
\AtBeginPart{
  \frame{\partpage}
}
\AtBeginSection{
  \ifbibliography
  \else
    \frame{\sectionpage}
  \fi
}
\AtBeginSubsection{
  \frame{\subsectionpage}
}
\usepackage{lmodern}
\usepackage{amssymb,amsmath}
\usepackage{ifxetex,ifluatex}
\ifnum 0\ifxetex 1\fi\ifluatex 1\fi=0 % if pdftex
  \usepackage[T1]{fontenc}
  \usepackage[utf8]{inputenc}
  \usepackage{textcomp} % provide euro and other symbols
\else % if luatex or xetex
  \usepackage{unicode-math}
  \defaultfontfeatures{Scale=MatchLowercase}
  \defaultfontfeatures[\rmfamily]{Ligatures=TeX,Scale=1}
\fi
% Use upquote if available, for straight quotes in verbatim environments
\IfFileExists{upquote.sty}{\usepackage{upquote}}{}
\IfFileExists{microtype.sty}{% use microtype if available
  \usepackage[]{microtype}
  \UseMicrotypeSet[protrusion]{basicmath} % disable protrusion for tt fonts
}{}
\makeatletter
\@ifundefined{KOMAClassName}{% if non-KOMA class
  \IfFileExists{parskip.sty}{%
    \usepackage{parskip}
  }{% else
    \setlength{\parindent}{0pt}
    \setlength{\parskip}{6pt plus 2pt minus 1pt}}
}{% if KOMA class
  \KOMAoptions{parskip=half}}
\makeatother
\usepackage{xcolor}
\IfFileExists{xurl.sty}{\usepackage{xurl}}{} % add URL line breaks if available
\IfFileExists{bookmark.sty}{\usepackage{bookmark}}{\usepackage{hyperref}}
\hypersetup{
  pdftitle={The Rise and Fall of Shadow Banking in China},
  pdfauthor={Mingze Huang},
  hidelinks,
  pdfcreator={LaTeX via pandoc}}
\urlstyle{same} % disable monospaced font for URLs
\newif\ifbibliography
\setlength{\emergencystretch}{3em} % prevent overfull lines
\providecommand{\tightlist}{%
  \setlength{\itemsep}{0pt}\setlength{\parskip}{0pt}}
\setcounter{secnumdepth}{-\maxdimen} % remove section numbering

\title{The Rise and Fall of Shadow Banking in China}
\subtitle{Market Signaling and Optimal Regulation}
\author{Mingze Huang}
\date{December 12, 2020}

\begin{document}
\frame{\titlepage}

\begin{frame}{Model}
\protect\hypertarget{model}{}

\begin{itemize}
\item
  Single-period economy
\item
  a (mass 1) continuum of banks and investors plus a government.
\item
  ``Safe assets'': pay \(y_s\) per unit of investment with probability
  \(p_s\), \(0\) otherwise.
\item
  A fraction of \(\alpha\) banks have access to superior risky assets
  pay \(y_{r}\) with probability \(p_{s}\); \(1-\alpha\) banks have
  access to inferior risky assets pay \(y_{r}\) with probability
  \(p_{r}\).
\item
  \(y_{r}>y_{s}\); \(p_{r}<p_{s}\).
\end{itemize}

\end{frame}

\begin{frame}{Assumption 1 (Assets' Payoff)}
\protect\hypertarget{assumption-1-assets-payoff}{}

\begin{itemize}
\item
  \(p_{s}y_{r}>p_{s}y_{s}>p_{r}y_{r}>1\) (superior risky assets pay more
  in expectation than safe assets, which pay in expectation more than
  inferior risky assets).
\item
  \(p_{s}y_{s}>\alpha p_{s}y_{r}+(1-\alpha)p_{r}y_{r}\) (safe assets pay
  more in expectation than risky assets).
\item
  \((1+\kappa)y_{r}>(1+\kappa)y_{s}>R\) (successful assets are enough to
  repay R).
\item
  \(p_{s}[(1+\kappa)y_{r}-R]>p_{r}[(1+\kappa)y_{r}-R]>p_{s}[(1+\kappa)y_{s}-R]\)
  (risk-shifting: including funding costs, it is more profitable for
  banks to invest always in risky assets).
\end{itemize}

\end{frame}

\begin{frame}{Assumption 2 (Information Structure)}
\protect\hypertarget{assumption-2-information-structure}{}

\begin{itemize}
\tightlist
\item
  Banks can distinguish all asset types, the government can just
  distinguish between safe and risky assets (not between superior and
  inferior risky assets), and investors cannot distinguish any asset
  type.
\end{itemize}

\end{frame}

\begin{frame}{Timing}
\protect\hypertarget{timing}{}

\begin{itemize}
\item
  At the beginning of period, each bank observes whether it has access
  to a superior or an inferior risky assets.
\item
  Then decides whether to raise funds by traditional banking or shadow
  banking, which offer interest rates \(R_{TB}\) or \(R_{SB}\).
\item
  Traditional banking subjects to capital constraint, it is forced to
  invest in safe assets.
\item
  Shadow banking allows bank chooses whether to invest in risky or safe
  assets.
\end{itemize}

\end{frame}

\begin{frame}{Traditional Banking}
\protect\hypertarget{traditional-banking}{}

\begin{itemize}
\tightlist
\item
  Capital requirement for traditional banking:
  \(\frac{\kappa}{\omega_{s}E(v_{s})}>\chi>\frac{\kappa}{\omega_{r}E(v_{r})}\)

  \begin{itemize}
  \tightlist
  \item
    \(E(v_{s})=(1+\kappa)p_{s}y_{s}\) is expected (market) value of safe
    assets
  \item
    \(E(v_{r})=(1+\kappa)(\alpha p_{s}+(1-\alpha)p_{r})y_{r}\) is
    expected (market) value of risky assets (as regulator cannot
    identify whether the risky asset is superior or inferior).
  \end{itemize}
\item
  Assume regulators choose \(\omega_{s}\), \(\omega_{r}\) and \(\chi\)
  (given \(\kappa\), \(E(v_{s}\) and \(E(v_{r})\)) to implement
  risk-weighted capital requirements such that traditional banking can
  operate only if it invests in safe assets.
\end{itemize}

\end{frame}

\begin{frame}{Traditional Banking}
\protect\hypertarget{traditional-banking-1}{}

\begin{itemize}
\item
  As banks have all bargaining power, investors' outside option is
  risk-free storage with no return. Then \(R^*_{TB}=\frac{1}{p_{s}}\).
\item
  Banks' profits in traditional banking are independent of risky asset
  type.
\end{itemize}

\[
\begin{align*}
\Pi_{TB}=&\kappa(p_{s}y_{s}-1)+p_{s}(y_{s}-R^*_{TB})\\
=&(1+\kappa)(p_{s}y_{s}-1)
\end{align*}
\]

\end{frame}

\begin{frame}{Shadow Banking}
\protect\hypertarget{shadow-banking}{}

\begin{itemize}
\item
  Capital requirement for shadow banking:
  \(\frac{\kappa}{\omega_{s}[E(v_{s}-R_{SB})]}>\frac{\kappa}{\omega_{r}[E(v_{r})-R_{SB}]}>\chi\).
\item
  Fraction of banks with access to superior risky assets that
  participate in shadow banking as:
  \(\chi_{s}\equiv\frac{\alpha\sigma_{F}(r_{s})}{\alpha\sigma_{F}(r_{s})+(1-\alpha)\sigma_{F}(r_{i})}\).
\item
  As Banks have all bargaining power:
  \(R^*_{SB}=\frac{1}{\chi_{s}p_{s}+(1-\chi_{s})p_{r}}\).
\item
  If only banks with access to superior risky assets participate in
  shadow banking, \(\chi_{s}=1\) and than
  \(R^*_{SB|\chi_{s}=1}=R^*_{TB}\).
\item
  If only banks with access to inferior risky assets participate in
  shadow banking, \(\chi_{s}=0\) and
  \(R^*_{SB|x_{s}=0}=\frac{1}{p_{r}}>R^*_{TB}\). This is why
  \(R_{SB}\in[\frac{1}{p_{s}}, \frac{1}{p_{r}}]\).
\end{itemize}

\end{frame}

\begin{frame}{Shadow Banking}
\protect\hypertarget{shadow-banking-1}{}

\begin{itemize}
\tightlist
\item
  Banks' profit in shadow banking (with superior risky assets \(r_{s}\),
  and inferior risky assets \(r_{i}\), respectively).
\end{itemize}

\[
\begin{align*}
\Pi_{SB}(r_{s})=&\kappa(p_{s}y_{r}-1)+p_{s}(y_{r}-R^*_{SB})\\
=&p_{s}[(1+\kappa)y_{r}-R^*_{SB}]-\kappa
\end{align*}
\]

\[
\Pi_{SB}(r_{i})=p_{r}[(1+\kappa)y_{r}-R^*_{SB}]-\kappa
\]

\end{frame}

\begin{frame}{Coexistence of Traditional and Shadow Banking}
\protect\hypertarget{coexistence-of-traditional-and-shadow-banking}{}

\begin{itemize}
\item
  Define \(\Delta^+\equiv(1+\kappa)[p_{s}y_{r}-p_{s}y_{s}]\) as
  additional expected payoffs of investing in superior risky assets
  relative to investing in safe assets.
\item
  Define \(\Delta^-\equiv(1+\kappa)[p_{s}y_{s}-p_{r}y_{r}]\) as
  additional expected payoffs of investing in safe assets relative to
  investing in inferior risky assets.
\item
  The equilibrium is characterized by: (i) all banks with access to
  superior risky assets finance in shadow banking
  (\(\sigma^*_{F}(r_{s})=1\)) and invest in those superior risky assets;
  and (ii) a fraction \(\sigma^*_{F}(r_{i})\in(0,1)\) of banks with
  access to inferior risky assets finances in shadow banking and invests
  in those assets, while the rest finance in traditional banking and
  invest in safe assets.
\end{itemize}

\end{frame}

\begin{frame}{Coexistence of Traditional and Shadow Banking}
\protect\hypertarget{coexistence-of-traditional-and-shadow-banking-1}{}

\begin{itemize}
\tightlist
\item
  Traditional banking interest rate is \(R^*_{TB}=\frac{1}{p_{s}}\) and
  shadow banking interest rate satisfies: \[
  \Delta^-=1-p_{r}R^*_{SB}(\sigma^*_{F}(r_{i}))
  \]
\end{itemize}

\end{frame}

\begin{frame}{Proof}
\protect\hypertarget{proof}{}

Consider first an equilibrium with shadow banking.

\begin{itemize}
\item
  There is no equilibrium in which \(\sigma_{F}(r_{s})>0\) and
  \(\sigma_{F}(r_{i})=0\). So that \(\chi_{s}=1\) and
  \(R_{TB}=R_{SB}=\frac{1}{p_{s}}\).

  \begin{itemize}
  \tightlist
  \item
    Bank with superior risky assets (\(r_{s}\)) do not deviate (indeed
    \(\sigma_{F}(r_{s})=1\)) because
  \end{itemize}
\end{itemize}

\[
\begin{align*}
\Pi_{SB}(r_{s})=&p_{s}[(1+\kappa)y_{r}-\frac{1}{p_{s}}]-\kappa\\ 
\Pi_{TB}=&p_{s}[(1+\kappa)y_{s}-\frac{1}{p_{s}}]-\kappa\\
\Pi_{SB}(r_{s})>&\Pi_{TB}\hspace{.2in}\text{as}\hspace{.2in}\Delta^+>0
\end{align*}
\]

\end{frame}

\begin{frame}{Proof}
\protect\hypertarget{proof-1}{}

\begin{itemize}
\tightlist
\item
  Bank with inferior risky assets (\(r_{i}\)) always deviate because
\end{itemize}

\[
\begin{align*}
\Pi_{SB}(r_{i})=&p_{r}[(1+\kappa)y_{r}-\frac{1}{p_{s}}]-\kappa\\
\Pi_{TB}=&p_{s}[(1+\kappa)y_{s}-\frac{1}{p_{s}}]-\kappa\\
\Pi_{SB}(r_{i})>&\Pi_{TB}
\end{align*}
\]

\begin{itemize}
\tightlist
\item
  According to risk-shifting Assumption 1, point (iv).
\item
  This implies that full separation is not an equilibrium because banks
  with access to inferior risky assets would rather raise funds in
  shadow banking at the same rates that apply in traditional banking and
  avoid regulation.
\end{itemize}

\end{frame}

\begin{frame}{Proof}
\protect\hypertarget{proof-2}{}

Then consider an equilibrium with shadow banking should involve the
participation of banks with inferior risky assets (\(r_{i}\)).

\begin{itemize}
\tightlist
\item
  As \(\sigma_{F}(r_{i})>0\), banks with superior risky assets
  (\(r_{s}\)) are always more profitable than banks with inferior risky
  assets (\(r_{i}\)) on shadow banking market because
\end{itemize}

\[
\begin{align*}
\Pi_{SB}(r_{s})=&\kappa(p_{s}y_{r}-1)+p_{s}(y_{r}-R^*_{SB})\\
=&p_{s}[(1+\kappa)y_{r}-R^*_{SB}]-\kappa\\
\Pi_{SB}(r_{i})=&p_{r}[(1+\kappa)y_{r}-R^*_{SB}]-\kappa\\
\Pi_{SB}(r_{s})>&\Pi_{SB}(r_{i})
\end{align*}
\] - According to Assumption 1, point (iv).

\begin{itemize}
\tightlist
\item
  Indeed \(\sigma^*_{F}(r_{s})=1\).
\end{itemize}

\end{frame}

\begin{frame}{Proof}
\protect\hypertarget{proof-3}{}

\begin{itemize}
\tightlist
\item
  The optimal strategy \(\sigma^*_{F}(r_{i})\) is determined by the
  point at which banks with inferior risky assets (\(r_{i}\)) weakly
  prefer to participate in shadow banking:
\end{itemize}

\[
\begin{align*}
\Pi_{SB}(r_{i}|\sigma^*_{F}(r_{i}))=&p_{r}[(1+\kappa)y_{r}-R^*_{SB}(\sigma^*_{F}(r_{i}))]-\kappa\\
\Pi_{TB}=&p_{s}[(1+\kappa)y_{s}-\frac{1}{p_{s}}]-\kappa\\
\Pi_{SB}(r_{i}|\sigma^*_{F}(r_{i}))\geq &\Pi_{TB}
\end{align*}
\]

or \[
R^*_{SB}(\sigma^*_{F}(r_{i}))\leq\frac{1-\Delta^-}{p_{r}}
\]

\end{frame}

\begin{frame}{Proof}
\protect\hypertarget{proof-4}{}

\begin{itemize}
\item
  In equilibrium, \(\sigma^*_{F}(r_{i})<1\). Otherwise
  \(R^*_{SB}(\sigma_{F}(r_{i})=1)=\frac{1}{\alpha p_{s}+(1-\alpha)p_{r}}\)
  and by Assumption 1, point (ii),
  \(\Delta^->1-p_{r}\frac{1}{\alpha p_{s}+(1-\alpha)p_{r}}\).
\item
  This implies that an equilibrium with pooling in shadow banking
  (hence, without traditional banking) does not exist. In words, if all
  banks with inferior risky assets (\(r_{i}\)) participate in shadow
  banking, the rate is so high that some banks with inferior risky
  assets (\(r_{i}\)) prefer to deviate and be regulated in traditional
  banking.
\end{itemize}

\end{frame}

\begin{frame}{Wealth Management Subsidiary}
\protect\hypertarget{wealth-management-subsidiary}{}

Suppose banks' type can be fully observed if they participate in shadow
banking through wealth management subsidiary.

\begin{itemize}
\tightlist
\item
  All banks with superior risky assets will participate in shadow
  banking through wealth management subsidiary because:
\end{itemize}

\[
\begin{align*}
R^W_{SB}(r_{s})=&\frac{1}{p_{s}}\\
\Pi^W_{SB}(r_{s})=&p_{s}[(1+\kappa)-\frac{1}{p_{s}}]-\kappa\\
\Pi_{SB}(r_{s})=&\kappa(p_{s}y_{r}-1)+p_{s}(y_{r}-R^*_{SB})\\
=&p_{s}[(1+\kappa)y_{r}-R^*_{SB}]-\kappa\\
\Pi_{TB}=&p_{s}[(1+\kappa)y_{s}-\frac{1}{p_{s}}]-\kappa
\end{align*}
\]

\end{frame}

\begin{frame}{Wealth Management Subsidiary}
\protect\hypertarget{wealth-management-subsidiary-1}{}

\[
\begin{align*}
\Pi^W_{SB}(r_{s})>&\Pi_{TB}\\
\Pi^W_{SB}(r_{s})\geq&\Pi_{SB}(r_{s})
\end{align*}
\]

\begin{itemize}
\tightlist
\item
  This implies that banks with superior risky assets (\(r_{s}\)) tend to
  signal their assets type on shadow banking market lower financing cost
  as return on risky assets is always higher than safe assets.
\end{itemize}

\end{frame}

\begin{frame}{Wealth Management Subsidiary}
\protect\hypertarget{wealth-management-subsidiary-2}{}

\begin{itemize}
\item
  Since there is no superior risky assets on shadow banking market
  outside wealth management channel, \(R^*_{SB}=\frac{1}{p_{r}}\).
\item
  All banks with inferior risky assets (\(r_{i}\)) would prefer to
  investing in safe assets and be regulated in traditional bank because
\end{itemize}

\[
\begin{align*}
\Pi_{SB}(r_{i})=&p_{r}[(1+\kappa)y_{r}-\frac{1}{p_{r}}]-\kappa\\
\Pi_{TB}=&p_{s}[(1+\kappa)y_{s}-\frac{1}{p_{s}}]-\kappa\\
\Pi_{SB}(r_{i})< &\Pi_{TB}
\end{align*}
\]

\end{frame}

\begin{frame}{Wealth Management Subsidiary}
\protect\hypertarget{wealth-management-subsidiary-3}{}

This implies that banks with inferior risky assets (\(r_{i}\)) would
exit shadow banking market due to high financing cost once the market
reveals their assets type.

\end{frame}

\end{document}
